% =============================================================================
% =  The Problem (1 page)
% =============================================================================

\section{Keeping Users Engaged Long Term} \label{sectionTheProblem}

% --- The problem is that some projects fail to evolve ------------------------

When developing a new system or a new software idea, getting the project off the ground and in front of users is one thing. However, keeping that project alive with a thriving community of engaged users is another.

The systems we create could be customer-facing web applications, games, or internal applications used to carry out tasks. Regardless of the type of system, the product will no longer provide usefulness without evolving with the user's needs. Even in a corporate setting with internal business systems, over time, users will need change; how a system can adapt to those needs requires a level of flexibility.

\vspace{0.25cm}
\begin{displayquote}
``Software evolution is the continual development of software after its initial release to address changing stakeholder and/or market requirements.'' \cite{wiki:software-evolution}
\end{displayquote}
\vspace{0.25cm}

% --- Why does it matter if a project does not evolve? ------------------------
\subsection{Why does software evolution matter?}

When a system cannot evolve, the impact is primarily felt by the users. However, this impact will eventually get back to those who created and continue to support the system. With users that are either unsatisfied or unable to use the system any longer, the engagement levels will drop. The decline in users will ultimately result in a loss of income, as the system can no longer deliver to the needs of its audience.

Because organizations invest large amounts of money in the software systems that they create, they depend on the software's continued success. Software evolution will allow the system to adapt to new or changing business requirements, fix bugs and defects, and integrate with other systems that have changed and evolved that may share the same software environment.

As a system is used, inevitably, users will stumble into situations that even the best quality assurance testers will miss. When defects are found, they will require fixing. 

To keep a system up-to-date, we must add new features. For example, there may be a need to improve a system's performance or reliability, especially if the user base expands.

Security can also impact the need for a system to be maintained. New ways to infiltrate a system can be uncovered, so it is important to stay on top of newest versions of dependencies and technologies in order to avoid potential breaches of data and experience.

% --- How do we ensure a project will be able to evolve? ----------------------
\subsection{How do we ensure software evolution?}

Because the maintainability of a system can ultimately influence the ability to generate revenue, we must find ways to ensure that a project will evolve. One of these ways could be to ensure that a project continues to be considered ``maintainable'' throughout its lifetime. This system characteristic will ensure that bugs can be fixed quickly, but new features should be easy to add as the users' needs evolve.
