% =============================================================================
% =  The Problem (1 page)
% =============================================================================

\section{Keeping Users Engaged Long Term} \label{sectionTheProblem}

% --- The problem is that some projects fail to evolve ------------------------

When developing a new system or a new software idea, getting the project off the ground and in front of users is one thing. However, keeping that project alive with a thriving community of engaged users is another.

The systems we create could be customer-facing web applications, games, or internal applications used to carry out tasks. Regardless of the system, the product will no longer provide usefulness without evolving with the user's needs. As users become familiar with a system that is designed to depend, at least in part, on their attitudes and their practices, the users will modify their behavior to minimize their effort or maximize the effectiveness of the tool \cite{lehman:1980}. Even in a corporate setting with internal business systems, the needs of users will change; how a system can adapt to those needs requires a level of flexibility.

% --- Why does it matter if a project does not evolve? ------------------------
\subsection{Why does software evolution matter?} \label{subWhySoftwareEvolution}

\vspace{0.25cm}
\begin{displayquote}
  ``Software evolution is the continual development of software after its initial release to address changing stakeholder and/or market requirements.'' \cite{wiki:software-evolution}
\end{displayquote}
\vspace{0.25cm}

When a system cannot evolve, it is mostly the users who feel the impact. However, this impact will eventually get back to those who created and continue to support the system. With users that are either unsatisfied or unable to use the system any longer, the engagement levels will drop. The decline in users will ultimately result in a loss of income, as the system can no longer deliver to the needs of its audience.

Because organizations invest large amounts of money in the software systems they create, they depend on their continued success. Software evolution will allow the system to adapt to new or changing business requirements, fix bugs and defects, and integrate with other systems that have changed and evolved that may share the same software environment.

To keep a system up-to-date, we must add new features. For example, there may be a need to significantly improve a system's performance or reliability if the user base expands. For a business or group to maintain user satisfaction, the software must continue to evolve; this will result in a system's increased size and complexity, as well as quality decline (unless very closely monitored over time) \cite{yu:2013}.

Security can also impact the need for a system to be maintained. Namely, nefarious people can uncover new ways to infiltrate a system, so it is vital to stay on top of the newest versions of dependencies and technologies to avoid potential breaches of data and experience.

% --- How do we ensure a project will be able to evolve? ----------------------
\subsection{How do we ensure software evolution?} \label{subEnsureEvolution}

Because the maintainability of a system can ultimately influence the ability to generate revenue, we must find ways to ensure that a project will evolve. One of these ways could be to ensure that a project continues to be considered ``maintainable'' throughout its lifetime. This system characteristic will ensure that we can fix bugs quickly, but new features should be easy to add as the users' needs evolve.

Not only is the user engagement necessary for the activation of the evolution process, but the engagement of the development community itself, especially when considering open source projects, is critical. As Adewumi et al. point out, a critical feature that distinguishes open source software is built and maintained by a community \cite{haaland:2010}. Additionally, the quality of the community will determine the quality of the open-source software \cite{samoladas:2008}.
