% =============================================================================
% =  Related Works
% =============================================================================

\section{Related Work} \label{sectionRelatedWork}

In our research, we are running with the assumption that Maintainability Index (MI) is our primary indicator and we will look for the correlation between the MI and other Pylint scores. We hope to find which correlations align and which are the most important.

\subsection{Considering Data Sets}

When exploring the correlations of maintainability and refactoring, there are many sources available for research. Some researchers have looked at proprietary systems as they evolve over time, while others have chosen open-source code available to the general public.

A study conducted by Baishakhi Ray, Daryl Posnett, Premkumar Devanbu, and Vladimir Filkov begins by programmatically collecting a sample set of projects in GitHub that vary in languages. Then the group of projects is appropriately culled, resulting in a final set used for the review. The results are then studied for the impact different programming languages may have on the code quality \cite{baishakhi:2017}. Through their research, they were able to determine which languages were more prone to defects, and that individual languages are more related to individual bugs rather than bugs overall.

The authors of ``Predicting Maintainability with Object-Oriented Metrics - An Empirical Comparison'' performed a very similar study to what we are doing here. Their study focuses on object-oriented software (specifically C/C++ and Java) and a correlation analaysis between object-oriented metrics and software maintainability, looking for the best metrics to predict maintainability \cite{janke:2003}. This particular study focuses on a few hand-picked software systems with an analysis of the change logs. Our study, however, will be of a larger scale (about 50 software systems) and focused solely on Python-heavy projects.

\subsection{Design Patterns and Software Quality}

In the paper ``Impact of design patterns on software quality: a systematic literature review'' the authors compared the use of design patterns to software evolution and maintainability. They found that design patterns provided clear flexibility when they reviewed changes that extended (evolved) software \cite{wedyan:2020}.

\vspace{0.25cm}
\begin{displayquote}
``Changes performed in a class can be corrective, adaptive, perfective, or preventive. These changes can occur due to new requirements, debugging, changes that propagate from changes in other classes and refactoring.''
\end{displayquote}
\vspace{0.25cm}

Wedyan and Abufakher found that there were two reasons that a class had more frequent changes \cite{wedyan:2020}:

\vspace{0.25cm}
\begin{enumerate}
    \item The class was easy to extend.
    \item The class correlated to other classes (raising alarms about class modularity).
\end{enumerate}
\vspace{0.25cm}

With these findings in mind, we intentionally aim to focus our research on changes for system extensions and adaptations rather than bug fixes that appeared to be larger change due to high coupling. Within this paper, we were able to focus on Refactor scores (code smells) rather than Error scores (bugs) within the system.

\subsection{Software Architecture and Maintainability}

The research done in ``Software Architecture Metrics: A Literature Review'', the authors discuss how early detection of issues within the software's architecture is key to mitigating the risk of poor performance and can lower the cost of repairing faults \cite{coulin:2019}. While most developers have had access to these types of metrics for several decades, the industry and open-source community have not really latched onto their use for keeping code in easy-to-work-with condition.

The review done by Coulin et al. called out five important qualities of software architecture \cite{coulin:2019}:

\vspace{0.25cm}
\begin{enumerate}
    \item Maintainability
    \item Extensibility
    \item Simplicity, Understandability
    \item Re-usability
    \item Performance
\end{enumerate}
\vspace{0.25cm}

Focusing on these qualities can narrow down the choice between different design options to end up with the most ideal solution. Keeping these five qualities in top-of-mind for new (and changed) code allows for easier future development and evolution of the software system.
