\section{Filtered Respository Set} \label{sectionFilteredSet}

Once we had a narrowed set of projects from GitHub that were primarily written in Python, we culled the set more using several criteria:

\vspace{0.25cm}
\begin{enumerate}
    \item Projects that are at least 80\% Python
    \item Projects with a long history of commits
    \item Projects with large development teams (community of contributors)
    \item Projects with many releases
    \item Projects of a substantial age
\end{enumerate}
\vspace{0.25cm}

Armed with this list, we were able to use the metadata from GitHub for each of our repositories already collected and determine a cross-section of these criteria that would result in about 50 repositories for futher study.

Beginning with the languages field from GitHub, we could easily narrow down projects that had at least 80\% of the code in Python. In our set, 103 repositories contained 80\% or more Python code.

With this narrowed set, we then looked to see at which percentile all the remaining criteria would yield the desired number of repositories. We determined that using the value at the 20th percentile in each of the above categories would yield the size set we'd need.

\begin{table}[ht]
  \centering
  \begin{tabularx}{0.8\textwidth} {
    | >{\centering\arraybackslash}X 
    | >{\centering\arraybackslash}X |
  }
    \hline
      Criteria & 20th Percentile Value \\ 
    \hline\hline
      Number of Commits & 2,968 \\
      Number of Contributors & 90 \\
      Number of Releases & 44 \\
      Age (in months) & 66.4 \\
    \hline
  \end{tabularx}
  \caption{Criteria used to filter down the initial set of repositories.}
  \label{table:repositoryPercentiles}
\end{table}

Table \ref{table:repositoryPercentiles} shows the values found for each of our criteria. Using these values as our minimum requirements, we can narrow our repository set to 46 repositories (see Table \ref{table:repositorySet}).

\begin{table}[ht]
  \centering
  \begin{tabularx}{1.0\textwidth} {
    | >{\centering\arraybackslash}X |
  }
    \hline
      Respository Names \\ 
    \hline\hline
      ansible
      astropy
      autobahn-python
      aws-cli
      beets
      biopython
      boto
      buildbot
      celery
      cobbler
      conda
      cython
      django
      django-rest-framework
      electrum
      fail2ban
      gensim
      luigi
      matplotlib
      mongoengine
      mitmproxy
      mongo-python-driver
      mopidy
      networkx
      paramiko
      nova
      numba
      pandas
      peewee
      pelican
      pip
      pyramid
      ranger
      raven-python
      salt
      scikit-image
      scikit-learn
      scrapy
      sentry
      sqlalchemy
      swift
      sympy
      tornado
      web2py
      werkzeug
      youtube-dl \\
    \hline
  \end{tabularx}
  \caption{List of the 46 repositories for research focus.}
  \label{table:repositorySet}
\end{table}
