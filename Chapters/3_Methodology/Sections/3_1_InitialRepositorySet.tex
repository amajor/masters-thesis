\section{Initial Repository Set} \label{sectionInitialSet}

We have established that we have a problem with projects that fail to evolve, resulting in a loss of revenue. We also understand that evolving software is essential to keep users engaged; without it, there is an appearance in the decline of quality, the program becomes less satisfactory to the user, and the potential for competitors to outpace us with features available. We must now understand how we can ensure that our systems evolve. For this, we will look to understand how the system's structural quality impacts software evolution.

The work done by Dr. Omari and Dr. Martinez involves collecting a sub-set of Python projects that we can use for further research. The bulk of their effort is to determine which classifiers to use to pare down the public set of Python systems into a good collection for further analysis \cite{omari:2018}. In addition, we have used the work they have provided to select appropriate Python systems for review by collecting meta-data on these code systems.

Selection criteria employed by Omari and Martinez included ``popular projects with long development history and multiple release cycles \cite{omari:2018}.'' All projects are open source, enabling other researchers to access the defined set. Additionally, they captured meta-data used to define our subset of repositories for our particular focus.

From their subset of repositories, we reviewed current Pylint scores from each of the 132 systems. This set gives us a sampling of data that we can now dig deeper into, comparing similar systems (similar size, a similar number of contributors, and more) and their evolution process by reviewing past commits rather than merely the system's current state here.
