\section{Initial Repository Set} \label{sectionInitialSet}

We have established that we have a problem with projects that fail to evolve, resulting in a loss of revenue. We also understand that evolving software is essential in order to keep users engaged; without it, there is an appearance in the decline of quality and the program becomes less satisfactory to the user, as well as potential for competitors to outpace us with features available. We must now understand how we can ensure that our systems evolve. For this, we will look to understand how the system's structural quality impacts software evolution.

The work done by Dr. Omari and Dr. Martinez involves collecting a sub-set of Python projects that we can use for further research. The bulk of the effort they have provided is determining which classifiers to use to pare down the public set of Python systems into a good collection for further analysis \cite{omari:2018}. The work that they have provided was used to select appropriate Python systems for review by collecting meta-data on these code systems.

From their subset of repositories, we were then able to collect current Pylint scores from each of our 129 systems. This gives us a sampling of data that we can now dig deeper into, comparing similar systems (similar size, similar number of contributors, etc.) and their evolution process by reviewing past commits rather than merely the current state of the system, as we have done here.
