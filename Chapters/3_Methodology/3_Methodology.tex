\chapter{Methodology} \label{chapterMethodology}

% In addition to the detailed methods you need to describe in this section, you need to provide specific objectives and an overview of your approach if they have not already been presented in the introductory chapters.  The best place to put those items can vary among theses.  Sometimes the background and lit review is really necessary to justify and substantiate the specific objectives and approach and, therefore, it is best to save those details for the beginning of this chapter.

There are a number of factors to consider when reviewing data and considering which software systems to consider. An obvious starting place is with open source software (OSS), as it is freely available to study.

To find projects of a caliber worth studying, we may also consider the maintenance capacity of a project.

\vspace{0.25cm}
\begin{displayquote}
  ``Maintenance capacity refers to the number of contributors to an OSS project and the amount of time they are willing and able to contribute to the development effort as observed from versioning logs, mailing lists, discussion forums and bug report systems. Furthermore, sustainability refers to the ability of the community to grow in terms of new contributors and to regenerate by attracting and engaging new members to take the place of those leaving the community.'' \cite{adewumi:2016}
\end{displayquote}
\vspace{0.25cm}
