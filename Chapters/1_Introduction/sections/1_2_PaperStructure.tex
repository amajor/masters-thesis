\section{Paper Structure}

% Chapter 2: Background & Literature Review    \ref{chapterBackground}
% 2.1 - Keeping users engaged long term        \ref{sectionTheProblem}
% 2.1.1 - Why software evolution matters
% 2.1.2 - How do we ensure software evolution
% 2.2 - The impact of structural quality       \ref{sectionMyIdea}
% 2.2.1 - Software maintenance
% 2.2.2 - Software evolution
% 2.2.3 - Measuring maintainability
% 2.2.4 - Maintainability scores
% 2.2.5 - Other maintainability characteristics
% 2.2.6 - Documentation and maintainability
% 2.4 - Related Work                           \ref{sectionRelatedWork}

In Chapter \ref{chapterBackground}, we will dig into a deeper background of the topic, exploring ideas of why software systems need to keep users engaged long term (Section \ref{sectionTheProblem}). We will explore automated measurements that provide evaluation scores of software systems. By using some of these quality and maintainability scores, we can see how structure impacts evolution (Section \ref{sectionMyIdea}). Additionally, we will explain how quality is measured and the different attributes that can factor into maintainability. We will also review related works (Section \ref{sectionRelatedWork}).

% Chapter 3: Methodology                       \ref{chapterMethodology}
% 3.1 - Initial repository set                 \ref{sectionInitialSet}
% 3.2 - Filtered repository set                \ref{sectionFilteredSet}

With more background on the problem, we can then review the methodology for our research in Chapter \ref{chapterMethodology}. Here we will review where we found our initial data set (Section \ref{sectionInitialSet}) and what criteria we used to filter it to a manageable size for our tests (Section \ref{sectionFilteredSet}).

% Chapter 4: Results                           \ref{chapterResults}

Chapter \ref{chapterResults} will review the results of our research using the methodology previously explained. We will explore the number of refactoring messages for each repository that we study relative to the size of the lines of code. Additionally, we will review the average Maintainability Index (MI) for each project reviewed. From there, we will compare the scores among the group of projects.

% Chapter 5: Conclusions and Recommendations   \ref{chapterConclusion}

We will then provide conclusions and recommendations in Chapter \ref{chapterConclusion}. In addition, the practices commonly considered the best to follow in computer programming are reinforced by much of our research, leading us to recommend well-known methods in automated delivery pipelines.
