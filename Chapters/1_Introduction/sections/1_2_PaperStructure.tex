\section{Paper Structure}

% Chapter 2: Background & Literature Review    \ref{chapterBackground}
% 2.1 - Keeping users engaged long term        \ref{sectionTheProblem}
% 2.1.1 - Why software evolution matters
% 2.1.2 - How do we ensure software evolution
% 2.2 - The impact of structural quality       \ref{sectionMyIdea}
% 2.2.1 - Software maintenance
% 2.2.2 - Software evolution
% 2.2.3 - Measuring maintainability
% 2.2.4 - Maintainability scores
% 2.2.5 - Other maintainability characteristics
% 2.2.6 - Documentation and maintainability
% 2.4 - Related Work                           \ref{sectionRelatedWork}

In Chapter \ref{chapterBackground}, we will dig into a deeper background of the topic, exploring ideas of why software systems need to keep users engaged long term (Section \ref{sectionTheProblem}). We will explore automated measurements that provide evaluation scores of software systems. By using some of these quality and maintainability scores, we can see how structure impacts evolution (Section \ref{sectionMyIdea}). Additionally, we'll explain how maintainability is measured, as well as the different attributes that can factor into maintainability. We will also review related works (Section \ref{sectionRelatedWork}).

% Chapter 3: Methodology                       \ref{chapterMethodology}
% 3.1 - Initial repository set                 \ref{sectionInitialSet}
% 3.2 - Filtered repository set                \ref{sectionFilteredSet}

With more background on the problem, we can then review the methodology for our research in Chapter \ref{chapterMethodology}. Here we will review where we found our initial data set (Section \ref{sectionInitialSet}) and what criteria we used to filter it to a manageable size for our tests (Section \ref{sectionFilteredSet}).

% Chapter 4: Results                           \ref{chapterResults}

Chapter \ref{chapterResults} will review the results of our research, using the methodology previously explained. \todo{TODO: Expand on this a little.}

% Chapter 5: Conclusions and Recommendations   \ref{chapterConclusion}

We will then provide final conclusions and recommendations in Chapter \ref{chapterConclusion}. \todo{TODO: Expand on this a little.}
