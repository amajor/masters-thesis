\section{Maintainability Index and Pylint Refactor Scores}

Pylint is a static analysis tool that identifies several classes of code quality concerns. Particularly relevant to our study are refactor violations, which report on various code smells. We can assume that there must be some correlation between Maintainability Index and the type and number of code smells in a software system, quantified by the Pylint refactor score.

This study explores such assumptions and systematically investigates any correlation between the Maintainability Index metric and the Pylint Refactor score. Furthermore, we perform analysis on specific refactor violations to reveal and shed light on the relative effectiveness of the different refactor violations and their relationship to Maintainability Index.

The structural quality of a software system will impact the software evolution. If the project has poor structural quality, the architecture will minimize its ability to evolve, and the software system will eventually ``die-off'' so to speak.

We will look at many open-source Python systems using Pylint and attempt to correlate the data from the Pylint scores to the level of ease in adding new features to the system. This will determine if a system is more maintainable with better Pylint scores.
