\chapter{Introduction} \label{sectionIntroduction}

% Chapter 1 - introduction into what I did

When building software systems, we have several areas of concern: cost, delivery timeline, quality, etc. The cost and time-to-market are often the two problems given the highest priority in a project. However, engineers must consider the software quality to preserve the system's longevity. Despite its importance, the code and architecture quality can be challenging to understand and measure.

When we think about projects, we can assume that as time goes on and changes and additions occur within a system's source code, the complexity of that system will grow. However, when we manage the code structure, we can keep the complexity in check, allowing systems to evolve. Developers can maintain this structure through simple steps like having readable code and more complex considerations, like how coupled and cohesive a system is.

One way to understand the quality around a system is to discuss its ``maintainability,'' the ease of receiving new features or resolving bugs. For example, developers may find that adjusting one area to add a new feature requires touching several other code areas in tightly coupled systems. Some code measuring systems provide a Maintainability Index (MI), a well-known quality measure. However, its effectiveness in quantifying software quality is debated \cite{vandeursen:2014}.

On the other hand, code smells are used extensively by practitioners to identify low-quality spots in the software system. These areas would need the teams' attention and are good candidates for refactoring.
