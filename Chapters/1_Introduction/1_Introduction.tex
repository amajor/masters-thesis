\chapter{Introduction} \label{sectionIntroduction}

The main goal of your introduction is to identify a problem that is worthy of investigation. It must also provide some idea of your research goals and approach to research.  Specific objectives can be introduced in the introduction chapter or they can be saved for later after you’ve provided additional background on the topic and state of the current research and its gaps.  The Introductory chapter often concludes with a summary of the organization of the thesis, including identification of the general content of specific chapters and appendices.

Ideally, chapter one defines the overall importance of the problem areas and provides an introduction into what you did, chapter two is why you did it in the context of what was previously known, three is how you did it, four is what you found and five is what it all means – putting the pieces together, (what’s your contribution to the research field).

It should be noted that the objectives of your research define the OUTCOME, i.e. what will be learned.  They are not a statement of the approach or tasks that are required to meet these objectives.  Some examples of reasonable research objectives:

\begin{itemize}
  \item Determine the effect of Marangoni convection on mixing of molten glasses
  \item Predict the extent of mechanical degradation of polymers
\end{itemize}

These both define the resulting outcome (prediction, effect on…) so they are objectives. The related tasks or research approach could be:

\begin{itemize}
  \item Solve a set of coupled non-linear PDEs\dots
  \item Perform experiments on\dots
\end{itemize}

These define the required steps; they do not define the outcome so they are NOT objectives.

Some theses and dissertations can have some chapters written as manuscripts that can be submitted to peer-reviewed scientific research journals. In that scenario, the grad student should be the principal author of the pending articles. The thesis or dissertation that includes manuscripts as chapters are not exempt from writing an introduction, background/ literature review and overall conclusions and recommendations.

This template uses the MS WORD STYLES extensively to help keep your work in the proper format.  These paragraphs use the “thesis-body text” style that is set for Times New Roman, 12 point font with double spaced lines and extra spacing between paragraphs (no need for hard carriage returns).  There are also styles for headers, equations, captions and bulleted lists that you can choose to use.  See examples throughout this template.
