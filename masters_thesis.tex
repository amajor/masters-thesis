\documentclass[12pt,letterpaper]{report}

% =============================================================================
% =  CUSTOM INFORMATION (stored as variables, used in template)
% =============================================================================

% The title of the paper.
\newcommand{\paperTitle}{Structural Quality \& Software Evolution}

% The name of the author.
\newcommand{\paperAuthor}{Alison Major}

% The author's concentration in their Master of Science in Computer Science
\newcommand{\authorConcentration}{Software Engineering}

% =============================================================================
% =  PREAMBLE (packages and custom code)
% =============================================================================

\usepackage{paperPreamble}

% =============================================================================
% =  BEGINNING OF THE DOCUMENT
% =============================================================================

\begin{document}

% Include Page Numbers - Roman Numerals for Front Matter
\cleardoublepage% \clearpage
\pagenumbering{roman}
\pagestyle{plain}

% =============================================================================
% =  Paper Title & Author Information
% =============================================================================

\input{Ancillary Pages/_title_page.tex}

% Set the document to be double spaced 
% http://kb.mit.edu/confluence/pages/viewpage.action?pageId=3907092
\doublespacing

% =============================================================================
% =  Signature Page
% =============================================================================

\newpage
The undersigned have examined the thesis entitled `\textbf{\paperTitle}' presented by \textbf{\paperAuthor}, a candidate for the degree of \textbf{Master of Science in Computer Science (Concentration in \authorConcentration)} and hereby certify that it is worthy of acceptance.

\todo{TODO: Add lines for signatures here. Justify above paragraph.}


% =============================================================================
% =  Paper Abstract
% =============================================================================

\newpage
\addcontentsline{toc}{section}{Abstract}
\section*{Abstract} \label{sectionAbstract}

% When writing an abstract, bare in mind an abstract is a short descriptive summary of your thesis. The number of words accepted might vary e.g. 200-250 words. An MS thesis abstract need not exceed two pages. Abstracts are typically written last although they are the most important part of the thesis. They should have a little bit of everything: the background, the scope of your project, the purpose, findings and conclusions. An abstract is neither paragraphed nor cited. It should not be written as a literature review or a discussion of results. In a simplistic manner, your abstract, in a few words, should answer the questions: why should we care about your research; how did you get your results; what did you learn, find, create, invent; and finally what do your results imply?

% State the problem.
Some software engineering projects fail to evolve, making them obsolete. When we build new software solutions, we consider several categories during planning: cost, time-to-deliver, and sometimes to a smaller extent, quality. Though quality can be a factor we consider when planning a project, it is a complex attribute to understand and measure.
% Say why this problem is interesting.
We care about quality, as it can impact the longevity of a project. Low-quality software can be complicated to enhance and evolve. Software that fails to evolve will fail to generate user engagement, leading to revenue loss.
Many tools are available for enforcing standards, some built into integrated development environments (IDEs) (like Visual Studio Code) and others as third-party linter tools. Two such tools are Pylint and Radon, which can analyze Python projects. In particular, we will focus on the refactor violations noted in Pylint and Radon's code reports, as these warnings can lead us to code smells.
% Say what my solution achieves.
We review some projects and resources to understand the correlation between software structure quality and its impacts on a system's evolving ability. For example, we find that our reviewed projects have a low count of refactoring messages and middle to high maintainability indexes, indicating that these measurements may help estimate the quality of a system.
% Say what follows from my solution.
With this understanding, we explore ways to improve the evolution of a software system through tools and suggestions.


% =============================================================================
% =  Acknowledgements
% =============================================================================

\newpage
\addcontentsline{toc}{section}{Acknowledgements}
\section*{Acknowledgements} \label{sectionAcknowledgements}

This research and working towards my degree has been an endeavor and one that I would not dream of tackling on my own.

Firstly, my thanks go to God, the ultimate creator, who formed us all to be creators fitting our skills.

Second, much gratitude goes to my very creative and patient family. I am thankful for my husband, Chris, who was foundational in finding the time and focus on working on my studies. My amazing kids, Ewan and Gwynnie, have provided patience, encouragement, and numerous trips to the library. I could not have gotten through this without all of your support.

Finally, I am grateful for my coworkers and my professors. The Robotic Process Automation (RPA) team at Sysco has been cheering me on from the sidelines. The professors at Lewis University have provided mentorship and guidance. They have all encouraged me as I've worked to apply my knowledge from the last decade of my experience as a software developer and gather new information that has helped us build better software solutions.


% =============================================================================
% =  Table of Contents
% =============================================================================

\newpage
\begin{singlespace}
  \tableofcontents
\end{singlespace}

% =============================================================================
% =  List of Tables
% =============================================================================

\newpage
\addcontentsline{toc}{section}{List of Tables}
\begin{singlespace}
  \listoftables
\end{singlespace}

% =============================================================================
% =  List of Figures
% =============================================================================

\newpage
\addcontentsline{toc}{section}{List of Figures}
\begin{singlespace}
  \listoffigures
\end{singlespace}

% =============================================================================
% =  Paper Content
% =============================================================================

% Include Page Numbers - Numbers (Arabic) for Main Matter
\cleardoublepage% \clearpage
\pagenumbering{arabic}

% \todo{========== INSERT SOMEWHERE ==========} 

% \begin{figure}[ht]
%   \centerline{
%     \includegraphics[width=0.7\columnwidth]{Lehman_Figure4.png}
%   }
%   \caption{Lehman's depiction of ``E-Programs'' \cite{lehman:1980}}
%   \label{figEPrograms}
% \end{figure}

% The programs we write become part of the world around that it models. These programs, especially with the advent of mobile technology, have become embedded in our world. Lehman recognized this decades ago and acknowledged the relationship of these types of programs to our world, as seen in Fig. \ref{figEPrograms}. \cite{lehman:1980}

% ``Software Maintainability is an indispensable factor to acclaim for the quality of particular software. It describes the ease to perform several maintenance activities to make a software adaptable to the modified environment. The availability \& growing popularity of a wide range of Machine Learning (ML) algorithms for data analysis further provides the motivation for predicting this maintainability.'' \cite{gupta:2021}

% ``This study would open new doors for the software developers for carrying out comparatively more precise predictions well in time and hence reduce the overall maintenance costs.'' \cite{gupta:2021}

% ``There are ten software maintainability measurements used in selected primary studies: CHANGE maintenance effort, corrective maintenance, adaptive maintenance effort, maintenance evaluation by maintainability index, maintenance evaluation by change proneness, maintenance time, maintenance cost, maintenance attributes, maintenance components and other measurements.'' \cite{alsolai:2019}

% There have been studies done to help visualize the evolution of software through the use of city maps. \cite{steinbruckner:2012}

% The EvoStreets approach follows 3 models to build up the map \cite{steinbruckner:2012}:
% \begin{enumerate}
%   \item The \textbf{primary model} for software cities captures structural and analysis data.
%   \begin{itemize}
%     \item Structural data refer to the static software structure, i.e. the decomposition into subsystems and modules, as well as dependencies among modules.
%   \end{itemize}
%   \item The secondary model is a 2.5D \textbf{geometric model}, representing primary model data
%   \begin{itemize}
%     \item captures the main structural and evolutionary properties of the software system
%     \item provides a specific, stable gestalt for each software system
%     \item --mainly influenced by the particular layout of the system elements
%   \end{itemize}
%   \item \textbf{Thematic tertiary models}
%   \begin{itemize}
%     \item designed to support specific application scenarios by visualizing particular aspects of a given software system and its development history
%   \end{itemize}
% \end{enumerate}

% ``Software process improvement (SPI) is seen as the dominant approach to improving software products in software development organisations [1].'' \cite{herranz:2019}

% % [1] --> Shih, C.-C., Huang, S.-J.: ‘Exploring the relationship between organizational culture and software process improvement deployment’, Inf. Manage., 2010, 47, (5–6), pp. 271–281

% ``SPI has become the primary approach to improving software quality and reliability, employee and customer satisfaction, and return on investment [2].'' \cite{herranz:2019}

% % [2] --> Mathiassen, L., Ngwenyama, O.K., Aaen, I.: ‘Managing change in software process improvement’, IEEE Softw.., 2005, 22, (6), pp. 84–91

% ``The result of this study shows that maintainability is the most significant and ubiquitous product quality characteristic considered in the literature while usability is the most significant attribute in the quality in use category.'' \cite{adewumi:2016}

% \todo{========== END THINGS TO INSERT ==========}

% Chapter one defines the overall importance of the problem areas and provides an introduction into what you did.
\newpage
\chapter{Introduction} \label{sectionIntroduction}

When building software systems, we have several areas of concern: cost, delivery timeline, quality, etc. The cost and time-to-market are often the two problems given the highest priority in a project. However, engineers must consider the software quality to preserve the system's longevity. Despite its importance, the code and architecture quality can be challenging to understand and measure.

When we think about projects, we can assume that as time goes on and changes and additions occur within a system's source code, the complexity of that system will grow. However, when we manage the code structure, we can keep the complexity in check, allowing systems to evolve. Developers can maintain this structure through simple steps like having readable code and more complex considerations, like how coupled and cohesive a system is.

One way to understand the quality around a system is to discuss its ``maintainability,'' the ease of receiving new features or resolving bugs. For example, developers may find that adjusting one area to add a new feature requires touching several other code areas in tightly coupled systems. Some code measuring systems provide a Maintainability Index (MI), a well-known quality measure. However, its effectiveness in quantifying software quality is debated \cite{vandeursen:2014}.

On the other hand, code smells are used extensively by practitioners to identify low-quality spots in the software system. These areas would need the teams' attention and are good candidates for refactoring.

\section{Maintainability Index and Pylint Refactor Scores}

``Software maintainability is one of the fundamental external quality attributes and is recognised as a research area of primary concern in software engineering \cite{alsolai:2019}.'' As such, it is important for us to find ways to measure the maintainability of a software system. As mentioned earlier, many OSS quality models already include maintainability as a characteristic measurement.

Pylint is a static analysis tool that identifies several classes of code quality concerns. Particularly relevant to our study are refactor violations, which report on various code smells. We can assume that there must be some correlation between Maintainability Index and the type and number of code smells in a software system, quantified by the Pylint refactor score.

This study explores such assumptions and systematically investigates any correlation between the Maintainability Index metric and the Pylint Refactor score. Furthermore, we perform analysis on specific refactor violations to reveal and shed light on the relative effectiveness of the different refactor violations and their relationship to Maintainability Index.

The structural quality of a software system will impact the software evolution. If the project has poor structural quality, the architecture will minimize its ability to evolve, and the software system will eventually ``die-off'' so to speak.

We will look at many open-source Python systems using Pylint and attempt to correlate the data from the Pylint scores to the level of ease in adding new features to the system. This will determine if a system is more maintainable with better Pylint scores.

\todo{TODO: Add info about Radon and MI and scores}

\section{Paper Structure}

% Chapter 2: Background & Literature Review    \ref{chapterBackground}
% 2.1 - Keeping users engaged long term        \ref{sectionTheProblem}
% 2.1.1 - Why software evolution matters
% 2.1.2 - How do we ensure software evolution
% 2.2 - The impact of structural quality       \ref{sectionMyIdea}
% 2.2.1 - Software maintenance
% 2.2.2 - Software evolution
% 2.2.3 - Measuring maintainability
% 2.2.4 - Maintainability scores
% 2.2.5 - Other maintainability characteristics
% 2.2.6 - Documentation and maintainability
% 2.4 - Related Work                           \ref{sectionRelatedWork}

In Chapter \ref{chapterBackground}, we will dig into a deeper background of the topic, exploring ideas of why software systems need to keep users engaged long term (Section \ref{sectionTheProblem}). We will explore automated measurements that provide evaluation scores of software systems. By using some of these quality and maintainability scores, we can see how structure impacts evolution (Section \ref{sectionMyIdea}). Additionally, we'll explain how maintainability is measured, as well as the difference attributes that can factor into maintainability. We will also review related works (Section \ref{sectionRelatedWork}).

% Chapter 3: Methodology                       \ref{chapterMethodology}
% 3.1 - Initial repository set                 \ref{sectionInitialSet}
% 3.2 - Filtered repository set                \ref{sectionFilteredSet}

With more background on the problem, we can then review the methodology for our research in Chapter \ref{chapterMethodology}. Here we will review where we found our initial data set (Section \ref{sectionInitialSet}) and what criteria we used to filter it to a manageable size for our tests (Section \ref{sectionFilteredSet}).

% Chapter 4: Results                           \ref{chapterResults}

Chapter \ref{chapterResults} will review the results of our research, using the methodology previously explained.

% Chapter 5: Conclusions and Recommendations   \ref{chapterConclusion}

We will then provide final conclusions and recommendations in Chapter \ref{chapterConclusion}.


% Chapter two is why you did it in the context of what was previously known.
\newpage
\chapter{Background and Literature Review} \label{chapterBackground}

% Should be about 20 pages or more.

% \todo{TODO: Chapter 2 - why I did it in the context of what was previously known}

% \todo{TODO: Provide sufficient fundamental background information about the subject to support my objectives, hypothesis (or research questions) and methods.}

% \todo{TODO: Review the pertinent literature related to the specific problem / hypothesis you are addressing.}

In this chapter, we will gain a better understanding of the topic at hand, first by understanding long-term user engagement. Then, we will explore the impact of structural quality within a software system. Once we understand the problem thoroughly, we will determine a valuable dataset to that we can apply our theories, as well as explore related works.

% =============================================================================
% =  The Problem (1 page)
% =============================================================================

\section{Keeping Users Engaged Long Term} \label{sectionTheProblem}

% --- The problem is that some projects fail to evolve ------------------------

When developing a new system or a new software idea, getting the project off the ground and in front of users is one thing. However, keeping that project alive with a thriving community of engaged users is another.

The systems we create could be customer-facing web applications, games, or internal applications used to carry out tasks. Regardless of the type of system, the product will no longer provide usefulness without evolving with the user's needs. Even in a corporate setting with internal business systems, over time, users will need change; how a system can adapt to those needs requires a level of flexibility.

\vspace{0.25cm}
\begin{displayquote}
``Software evolution is the continual development of software after its initial release to address changing stakeholder and/or market requirements.'' \cite{wiki:software-evolution}
\end{displayquote}
\vspace{0.25cm}

% --- Why does it matter if a project does not evolve? ------------------------
\subsection{Why does software evolution matter?}

When a system cannot evolve, the impact is primarily felt by the users. However, this impact will eventually get back to those who created and continue to support the system. With users that are either unsatisfied or unable to use the system any longer, the engagement levels will drop. The decline in users will ultimately result in a loss of income, as the system can no longer deliver to the needs of its audience.

Because organizations invest large amounts of money in the software systems that they create, they depend on the software's continued success. Software evolution will allow the system to adapt to new or changing business requirements, fix bugs and defects, and integrate with other systems that have changed and evolved that may share the same software environment.

As a system is used, inevitably, users will stumble into situations that even the best quality assurance testers will miss. When defects are found, they will require fixing. 

To keep a system up-to-date, we must add new features. For example, there may be a need to improve a system's performance or reliability, especially if the user base expands.

Security can also impact the need for a system to be maintained. New ways to infiltrate a system can be uncovered, so it is important to stay on top of newest versions of dependencies and technologies in order to avoid potential breaches of data and experience.

% --- How do we ensure a project will be able to evolve? ----------------------
\subsection{How do we ensure software evolution?}

Because the maintainability of a system can ultimately influence the ability to generate revenue, we must find ways to ensure that a project will evolve. One of these ways could be to ensure that a project continues to be considered ``maintainable'' throughout its lifetime. This system characteristic will ensure that bugs can be fixed quickly, but new features should be easy to add as the users' needs evolve.

% =============================================================================
% =  My Idea (2 pages)
% =============================================================================

\section{The Impact of Structural Quality} \label{sectionMyIdea}

% --- Understand Software Maintenance -----------------------------------------
\subsection{Software Maintenance} \label{subSoftwareMaintenance}

The structural quality of a software system will impact the software evolution. Similarly, as a project evolves, there is a likelihood that it will degrade, as changes are often made quickly and in ways that the original design did not anticipate \cite{martin:2000}. If the project has poor structural quality, its ability to evolve will be minimized, and the software system will eventually ``die-off'' so to speak.

There is much planning involved in all software creation projects in what the product will be, will do, who it is for, etc. One of the things that should also be on the planning list is long-term maintenance and growth. That is, how do we build a thing that will be easier to add features to down the road?

Let us define maintainability in the context of software. For example, a system would be considered easy to maintain if it is easy to debug and easy to add new features. These new features are generally considered minor features, and may often be reported as bugs by users, when in reality, they are looking for functionality enhancements \cite{wiki:software-maintenance}.

\vspace{0.25cm}
\begin{displayquote}
  ``Software maintenance in software engineering is the modification of a software product after delivery to correct faults, to improve performance or other attributes.'' \cite{wiki:software-maintenance}
\end{displayquote}
\vspace{0.25cm}

It may be easier to understand what characteristics define a system with poor maintainability. These types of systems will have poor code quality, leading to defects. For example, there could be undetected vulnerabilities or vulnerabilities that have been ignored. It may be that the system is overly complex. In addition to the complexity, it could be hard to read due to poor naming or dead (unused) code throughout the source code.

A project is known to have good maintainability when there is an enforced set of clean and consistent standards for the code. This often involves having human-readable names for functions, methods, and variables. Any complex code is minimized, and methods are small and focus on a single thing. Parts of the system are decoupled and organized, making it easy to work on different parts with low impact on unrelated parts. For example, the code is DRY (there is limited redundancy in the code), unused code has been removed, and there is a level of documentation that supports an easy understanding of the system.

Why should we care about whether the code is maintainable? It is assumed that a large amount of the cost over the lifetime of a project is attributed to maintainability. Fred Brooks, in his book ``The Mythical Man-Month'' (1975) even claimed that over 90\% of the costs for a typical software system come up in the maintenance phase \cite{brooks:mythical}. In 1977, Meir M. Lehman noted that 70\% of a program budget was spent on maintenance, with the remaining 30\% spent on development. In 1993, it was again observed that only 20-40\% of the resources (money, time, effort) were used for development of a project, with 60-70\% used for maintenance activities \cite{ieee:1219}.

Once the bulk of the system is off the ground and live worldwide, how well the team can improve the system with new features and fix bugs, even working on different parts in parallel, can be impacted by its maintainability. Any successful piece of software will inevitably need to be maintained.

\todo{TODO: Anything about SOLID and when too many files are being touched? \cite{martin:2000}}

% --- Understand Software Evolution -------------------------------------------
\subsection{Software Evolution} \label{subSoftwareEvolution}

There is a distinction to be made between \textbf{software maintenance} and \textbf{software evolution}. We will refer to software maintenance as bug resolution and for minor functional improvements. For example, we can consider this routine maintenance when we must fix a broken route in the application or provide a subtle enhancement on the user experience. However, when we look at upgrades to the system, adaptations to the changing and growing needs of the user, or migrating the system to a new technology, we can refer to this as evolution of the software.

The evolution of software can result from new laws that have come into being. As technology itself changes, governing bodies must continually revisit data collection and information sharing policies. Changes in technology and laws may lead to adaptations in the software systems.

It is also fair to say that systems will change because we can never fully determine a user's needs at the start of a project. It would be safe to say that the user's needs will change over time themselves. This leads to a never-ending project that will always need some form of enhancement.

Meir ``Manny'' Lehman and László ``Les'' Bélády contributed to a list of laws involving software evolution known as Lehman's Laws that describe a balance between forces that drive new developments while also slowing progress. These laws apply to programs that were written to perform some real-world activity, where its behavior is linked to the environment in which it runs; additionally, this program category assumes that the program needs to adapt to varying requirements and circumstances in that environment. Eight laws were created and are listed below. \cite{wiki:lehmans-laws}

\todo{TODO: Read and reference ``An Empirical Study of Lehman's Law on Software Quality Evolution'' \cite{yu:2013}}.

\vspace{0.25cm}
\begin{enumerate}
    % "Continuing Change" — an E-type system must be continually adapted or it becomes progressively less satisfactory.
    \item \textbf{Continuing Change} \textit{(1974)}
    
    % "Increasing Complexity" — as an E-type system evolves, its complexity increases unless work is done to maintain or reduce it.
    \item \textbf{Increasing Complexity} \textit{(1974)}

    % "Self Regulation" — E-type system evolution processes are self-regulating with the distribution of product and process measures close to normal.
    \item \textbf{Self Regulation} \textit{(1974)}

    % "Conservation of Organisational Stability (invariant work rate)" — the average effective global activity rate in an evolving E-type system is invariant over the product's lifetime.
    \item \textbf{Conservation of Organisational Stability} \textit{(1978)}

    % "Conservation of Familiarity" — as an E-type system evolves, all associated with it, developers, sales personnel, and users, for example, must maintain mastery of its content and behavior to achieve satisfactory evolution. Excessive growth diminishes that mastery. Hence the average incremental growth remains invariant as the system evolves.
    \item \textbf{Conservation of Familiarity} \textit{(1978)}

    % "Continuing Growth" — the functional content of an E-type system must be continually increased to maintain user satisfaction over its lifetime.
    \item \textbf{Continuing Growth} \textit{(1991)}

    % "Declining Quality" — the quality of an E-type system will appear to be declining unless it is rigorously maintained and adapted to operational environment changes.
    \item \textbf{Declining Quality} \textit{(1996)}

    % "Feedback System" (first stated 1974, formalised as law 1996) — E-type evolution processes constitute multi-level, multi-loop, multi-agent feedback systems and must be treated as such to achieve significant improvement over any reasonable base.
    \item \textbf{Feedback System} \textit{(1996)}
\end{enumerate}
\vspace{0.25cm}

The first law, ``Continuing Change,'' tells us that if a system does not adapt, it will become progressively less satisfactory. The second, ``Increasing Complexity,'' explains that as a system evolves, unless work is done to maintain or reduce complexity, the complexity will increase. This can be due to the added volume of the code from new features or even an increasing number of developers that have edited the code. Unless this phenomenon of increased complexity is actively addressed during changes, it can impact the maintainability (and the ability of a project to continue evolving) in the future.

Lehman's fifth law, ``Conservation of Familiarity,'' explains how the average incremental growth does not change over time as a system evolves. The people interacting with the system, such as the developers, business persons, or users, must still continue using and working within the system at the same ``level of mastery.'' If the system grows and changes excessively, the mastery will drop, slowing down the next set of changes. This could be because the source code or architecture has become more complex (impacting the developers' ability to adapt and enhance the system) or because the user features have changed so that the system audience needs time to master the new interfaces or new tools. Because of this natural ``slow-down'' for excessive change, the average incremental growth will remain steady. We can see a simplified visual in ``Fig.~\ref{figConservationOfFamiliarity}'' showing that when the number of changes spikes (that is to say, when there is excessive growth in a system), it will be followed by an iteration of fewer changes, leading to a nearly consistent average of incremental growth (the thick, horizontal line) over time.

\begin{figure}[ht]
    \centerline{
        \includegraphics[width=\columnwidth]{Changes-vs-Time}
    }
    \caption{A simplified visual of Lehman's fifth law, ``Conservation of Familiarity.''}
    \label{figConservationOfFamiliarity}
\end{figure}

In Lehman's sixth law, ``Continuing Growth,'' we see that the system user's satisfaction will not be maintained without continually increasing the functional content. Along a similar idea, the law pertaining to ``Declining Quality'' states that if the operational environment for the system does not change, the system's quality will appear to decline. Yu and Mishra performed focused research on supporting Lehman's Laws, especially in the case of the seventh law pertaining to declining quality, by defining a metric for software quality and reviewing the bug history, growth of size, complexity, and quality of two large open-source systems \cite{yu:2013}. Therefore, we must continue adapting for even the appearance of the maintained quality of a system.

With all of these characteristics surrounding the evolution of software, we benefit from the Internet that has positively improved the experience. Two common resources currently available to developers have impacted software evolution \cite{wiki:software-evolution}:

\vspace{0.25cm}
\begin{enumerate}
    \item The rapid growth of the World Wide Web and Internet Resources make it easier for users and engineers to find related information.
    \item Open source development where anybody could download the source codes and modify it has enabled fast and parallel evolution (through forks).
\end{enumerate}
\vspace{0.25cm}

These two suggestions are very evident in modern development. For example, a developer may regularly use resources like StackOverflow to find solutions to problems and use open-source tools that the developer and their team can contribute to or adjust to their specific needs.

\subsection{Measuring Maintainability} \label{subMeasureMaintainability}

Despite the nuanced differences between \textit{maintainability} and \textit{evolution}, the two characteristics run parallel to each other. If a system is easy to maintain, it will also be easier to evolve. If we can measure our system's maintainability, we can also determine if our system is in a good position to continue evolving to meet our future needs.

\vspace{0.25cm}
\begin{displayquote}
``The unit cost of change must initially be made as low as possible and its growth, as the system ages, minimized. Programs must be made more alterable, and the alterability maintained throughout their lifetime. The change process itself must be planned and controlled.'' \cite{lehman:1980}
\end{displayquote}
\vspace{0.25cm}

Several tools attempt to provide some value around these ideas. In this paper, we will focus on the metrics that Pylint provides, specifically looking into the Refactor score of Pylint.

We will look at many open-source Python systems using Pylint and attempt to correlate the data from the Pylint scores to the level of ease in adding new features to the system. This will determine if a system is more maintainable with better Pylint scores. 

\todo{FUTURE EDITION: To do this, we will measure the locality of the changes by the number of files that are edited in a commit. We will also focus on commits that represent new features, not on commits that are bug fixes.}

% -----------------------------------------------------------------------------
% -  (B) We can use maintainability scores to see how structure impacts evolution.
% -----------------------------------------------------------------------------
\subsection{Maintainability Scores} \label{subMaintainabilityScores}

% --- Claim B: Review each claim from the introduction

% \todo{TODO: What is the equation for maintainability index?}

First, let us consider our original understanding of software maintainability. While this definition focuses primarily on bug fixes and minor enhancements, maintainable projects should also have ease in their ability to evolve. Therefore, we can study the impact maintainability (structural quality) has on software evolution by reviewing the scores provided by automated code review tools.

In this study, we will be using Pylint and will focus on the values of the Refactor score regarding a set of open-source Python systems. We must understand what Pylint itself is doing to understand the scores we will be working with, 

Through the documentation of Pylint, we can understand how to use it and the scores it will provide \cite{pylint:main}. The Pylint score itself is calculated by the following equation \cite{pylint:score}:

\vspace{0.25cm}
\begin{center}
\code{10.0 - (( float( 5 * e + w + r + c) / s ) * 10 )}
\end{center}
\vspace{0.25cm}

Numbers closer to \code{10} reflect systems that have fewer errors, fewer warnings, and overall better structure and consistency. In the above equation, we are using the following values \cite{pylint:docs}:

\vspace{0.25cm}
\begin{itemize}
    \item \textbf{statement} (\code{s}): the total number of statements analyzed
    \item \textbf{error} (\code{e}): the total number of errors, which are likely bugs in the code
    \item \textbf{warning} (\code{w}): the total number of warnings, which are python specific problems
    \item \textbf{refactor} (\code{r}): the total number of refactor warnings for bad code smells
    \item \textbf{convention} (\code{c}): the total number of convention warnings for programming standard violations
\end{itemize}
\vspace{0.25cm}

The Refactor score is of particular interest to us and considers many features meticulously outlined on the Pylint site \cite{pylint:refactor}. These types of warnings include many checks, like prompting when to simplify a boolean condition, a useless \code{return}, and more. This score, in particular, will be part of our focus.

Pylint will check the code for code smells based on the definitions for documented checks to calculate the Refactor score. For every infraction, the score increases by one count. 

% -- How Pylint Refactor errors are related to architecture smells ------------

\vspace{0.25cm}
\begin{displayquote}
  ``In computer programming, a \textbf{code smell} is any characteristic in the source code of a program that possibly indicates a deeper problem.'' \cite{wiki:code-smells}
\end{displayquote}
\vspace{0.25cm}

% \todo{TODO: expand on this idea more}
We can use these Refactor scores to help us spot architecture smells. After all, code smells can point the way to deeper problems in our system. However, there are fundamental established design principles that we should consider when creating software; code smells alert us to areas that have deviated from these principles. These smells are drivers for refactoring and when addressed, can help us maintain the integrity of our architecture rather than creating a patchwork construction. Because of the relation of refactoring scores to the code structure itself, we will be spending much of our focus on this particular value.

Finally, concerning Python, it is also helpful to be familiar with PEP 8, as this is the default set of standards that Pylint uses to judge Python code \cite{pylint:pep8}. This standard leads to making code more readable and more consistent, which may contribute to the code being more maintainable than without the standards. These standards cover things like indentation spacing, maximum line length, where to insert new lines, how to handle imports, and more. By defining a set of standards, teams can ensure they have a defined set of rules so that any contributors to the code understand the expectations (and so that automated systems like Pylint can enforce those standards to maintain readability and consistency).

% --- Claim B: Identify the evidence (analysis and comparison, theorems, measurements, case studies)

\subsection{Other Maintainability Characteristics} \label{subOtherCharacteristics}

The authors of ``Measurement and refactoring for package structure based on complex network'' recently reviewed a similar idea focusing on cohesion and coupling over time for a project \cite{zhou:2020}. In a software system, we desire low coupling (allowing for changes to one area to remain independent of changes to another area) and high cohesion (indicating reduced complexity in modules, which improves maintainability). Through a few experiments on open-source software systems, the authors determined that their algorithm that calculated metrics could improve package structures to have high cohesion and low coupling. Their study gives us confidence that metrics around the software's structure can provide value in keeping systems in a maintainable state, which allows for software evolution.

Another variable that may impact the maintainability of code is readability. For example, in the article ``How does code readability change during software evolution?'' the authors have addressed this concern and found that most source codes were readable within the sample they reviewed. Additionally, a minority of commits changed the readability; if a file is less readable, it was likely that it remained that way and did not improve \cite{piantadosi:2020}. This variable (readability) in the maintainability of a software system can influence how easy or difficult it is to make a change. The authors also found that big commits, usually associated with adaptive changes (a form of software evolution), were the most prone to reduce code readability \cite{piantadosi:2020}. We assume that smaller commits are almost always better and can lead to more readable code.

Piantadosi et al. found that changes in readability, whether improvements or disintegrations, often occurred unintentionally \cite{piantadosi:2020}. By enforcing the PEP 8 standard, we know that Pylint is encouraging systems to remain readable. Therefore, projects that use some form of automated system in their pipeline benefit from keeping their project on track, limiting the effects of readability on a software's potential for evolution.

The paper ``Standardized code quality benchmarking for improving software maintainability'' provides insights into how the code's maintainability is impacted by the technical quality of source code \cite{baggen:2012}. Within their paper, the authors seek to show four key points: (1) how easy it is to determine where and how the change is made, (2) how easy it is to implement the change, (3) how easy it is to avoid unexpected effects, and (4) how easy it is to validate the changes. Their approach has shown that some tools and methods can improve and maintain technical quality within their projects, allowing systems to continue to evolve at a reasonable pace.

% -----------------------------------------------------------------------------
% -  (C) Documentation can improve maintainability.
% -----------------------------------------------------------------------------
\subsection{Documentation and Maintainability} \label{subDocumentation}

% --- Claim C: Review each claim from the introduction

Our assumption is that the Refactor score in projects should correlate to the evolution of the system. The first pass through the data is not conclusive in this particular detail, as the projects reviewed have many other factors contributing to the evolution of the project (number of contributors, size of the code system, etc.). Our assumption is that the correlation between software quality and software evolution would indicate that the better-scoring code systems are readable in themselves. In addition, it would be helpful to understand whether there are any similarities in how a system is documented that could contribute to improved software evolution of a system.

% \todo{TODO: what kind of documentation do the ``good'' projects have?}

% \todo{TODO: what kind of documentation do the ``bad'' projects have?}

% --- Claim C: Identify the evidence (analysis and comparison, theorems, measurements, case studies)

The textbook, ``Software Architecture in Practice,'' chapter 18 provides some insight in documentation around architecture \cite{book:software-architecture-in-practice}:

\vspace{0.25cm}
\begin{displayquote}
``If you go to the trouble of creating a strong architecture, one that you expect to stand the test of time, then you \textit{must} go to the trouble of describing it in enough detail, without ambiguity, and organizing it so that others can quickly find and update the needed information.''
\end{displayquote}
\vspace{0.25cm}

The book describes how documentation holds the results of significant design decisions, providing valuable insights into decisions down the road. While not directly related to the Pylint Refactor score and not within the source code itself, it is still helpful to remind ourselves that documentation can also influence the ability of a software system to evolve.

\todo{Our ``best scores'' (regarding the current Pylint Refactor score) were found to have relatively organized and useful documentation. The code repository for \emph{Munki} provided documentation for previous versions, lending insight into design decisions as the software evolved \cite{data:munki}. The repository for \emph{Raven}, however, was a deprecated version that has since been replaced by a paid platform known as \emph{Sentry}, but had ample documentation \cite{data:raven-python}. It is possible that the ``death'' of that software system was not lack of evolution, but rather a business decision. \emph{ElastAlert} was another system with good scores and easy-to-follow documentation, though it is focused more for the use of the system rather than how to enhance the system itself \cite{data:elastalert}.}

\todo{When reviewing our ``worst offenders'' in current Refactor scores, it was noted that even with poor scores, these repositories were able to continue to see engagement from developers. While further inspection will be needed to understand whether the code itself is evolving or just has engagement from a maintenance level, it is interesting to note that there is decent documentation provided. \emph{SymPy} goes as far as documenting the architecture for the software as well as design decisions, enabling developers to better understand the structure as they make contributions \cite{data:sympy-docs}.}

\vspace{0.25cm}
\begin{displayquote}
``Our study has shown that the primary studies provide empirical evidence on the positive effect of documentation of designs pattern instances on programme comprehension, and therefore, maintainability.''
\end{displayquote}

\begin{displayquote}
``...developers should pay more effort to add such documentation, even if in the form of simple comments in the source code.''
\end{displayquote}
\vspace{0.25cm}

In research done by Wedyan and Abufakher (quoted above), it was found that documenting design patterns was useful in enhancing code understanding \cite{wedyan:2020}. In turn, the comprehensibility impacts the maintainability of the code in a positive way, which continues to reinforce the impact that documentation can have and how it ties well into considerations for software structure.

% =============================================================================
% =  Related Works
% =============================================================================

\section{Related Work} \label{sectionRelatedWork}

In our research, we are running with the assumption that Maintainability Index (MI) is our primary indicator. Therefore, we will look for the correlation between the MI and other Pylint scores. We hope to find which correlations align and which are the most important.

\subsection{Considering Data Sets}

When exploring the correlations between maintainability and refactoring, many sources are available for research. For example, some researchers have looked at proprietary systems as they evolve, while others have chosen open-source code available to the general public.

A study conducted by Baishakhi Ray, Daryl Posnett, Premkumar Devanbu, and Vladimir Filkov begins by programmatically collecting a sample set of projects on GitHub that vary in languages. Then the group of projects is appropriately thinned out, resulting in a final set used for the review. The results are then studied for the impact different programming languages may have on the code quality \cite{baishakhi:2017}. Finally, their research determined which languages were more prone to defects and which individual languages were more related to individual bugs rather than bugs overall.

The authors of ``Predicting Maintainability with Object-Oriented Metrics - An Empirical Comparison'' performed a similar study to what we are doing here. Their study focuses on object-oriented software (specifically C/C++ and Java) and correlation analysis between object-oriented metrics and software maintainability. Janke et al. looked for the best metrics to predict maintainability \cite{janke:2003}. This study focuses on a few hand-picked software systems with an analysis of the changelogs. Our study, however, will be of a larger scale (about 50 software systems) and focused solely on Python-heavy projects.

\subsection{Design Patterns and Software Quality}

In the paper ``Impact of design patterns on software quality: a systematic literature review'' the authors compared the use of design patterns to software evolution and maintainability. They found that design patterns provided flexibility when reviewing changes that extended (evolved) software \cite{wedyan:2020}.

\vspace{0.25cm}
\begin{displayquote}
  ``Changes performed in a class can be corrective, adaptive, perfective, or preventive. These changes can occur due to new requirements, debugging, changes that propagate from changes in other classes and refactoring.''
\end{displayquote}
\vspace{0.25cm}

Wedyan and Abufakher found that there were two reasons that a class had more frequent changes \cite{wedyan:2020}:

\vspace{0.25cm}
\begin{enumerate}
    \item The class was easy to extend.
    \item The class correlated to other classes (raising alarms about class modularity).
\end{enumerate}
\vspace{0.25cm}

With these findings in mind, we intentionally aim to focus our research on changes for system extensions and adaptations rather than bug fixes that appeared to be more considerable change due to high coupling. This paper focused on Refactor scores (code smells) rather than Error scores (bugs) within the system.

\subsection{Software Architecture and Maintainability}

In the research done in ``Software Architecture Metrics: A Literature Review'', the authors discuss how early detection of issues within the software's architecture is key to mitigating the risk of poor performance and can lower the cost of repairing faults \cite{coulin:2019}. While most developers have had access to these metrics for several decades, the industry and open-source community have not latched onto their use for keeping code in easy-to-work-with conditions.

The review done by Coulin et al. called out five essential qualities of software architecture \cite{coulin:2019}:

\vspace{0.25cm}
\begin{enumerate}
    \item Maintainability
    \item Extensibility
    \item Simplicity, Understandability
    \item Re-usability
    \item Performance
\end{enumerate}
\vspace{0.25cm}

Focusing on these qualities can narrow down the choice between different design options to an ideal solution. Keeping these five qualities in top-of-mind for new (and changed) code allows for easier future development and evolution of the software system.

ISO/IEC 25010:2011 is a detailed standard for software quality that contains eight product quality characteristics  \cite{iso/iec:25010:2011}. Each characteristic is further comprised of various sub-characteristics. ``Fig.~\ref{figProductQualityModel}'' lists these characteristics, with maintainability being one of the most significant characteristics in some studies \cite{gupta:2021}, \cite{adewumi:2016}.

\begin{figure}[ht]
  \centerline{
    \includegraphics[width=0.7\columnwidth]{ProductQualityModel.png}
  }
  \caption{The eight product quality characteristics and sub-characteristics.}
  \label{figProductQualityModel}
\end{figure}


% Chapter three is how you did it.
\newpage
\chapter{Methodology} \label{chapterMethodology}

% In addition to the detailed methods you need to describe in this section, you need to provide specific objectives and an overview of your approach if they have not already been presented in the introductory chapters.  The best place to put those items can vary among theses.  Sometimes the background and lit review is really necessary to justify and substantiate the specific objectives and approach and, therefore, it is best to save those details for the beginning of this chapter.

There are a number of factors to consider when reviewing data and considering which software systems to consider. An obvious starting place is with open source software (OSS), as it is freely available to study.

To find projects of a caliber worth studying, we may also consider the maintenance capacity of a project.

\vspace{0.25cm}
\begin{displayquote}
  ``Maintenance capacity refers to the number of contributors to an OSS project and the amount of time they are willing and able to contribute to the development effort as observed from versioning logs, mailing lists, discussion forums and bug report systems. Furthermore, sustainability refers to the ability of the community to grow in terms of new contributors and to regenerate by attracting and engaging new members to take the place of those leaving the community.'' \cite{adewumi:2016}
\end{displayquote}
\vspace{0.25cm}

\section{Initial Repository Set} \label{sectionInitialSet}

We have established that we have a problem with projects that fail to evolve, resulting in a loss of revenue. We also understand that evolving software is essential to keep users engaged; without it, there is an appearance in the decline of quality, the program becomes less satisfactory to the user, and the potential for competitors to outpace us with features available. We must now understand how we can ensure that our systems evolve. For this, we will look to understand how the system's structural quality impacts software evolution.

The work done by Dr. Omari and Dr. Martinez involves collecting a sub-set of Python projects that we can use for further research. The bulk of their effort is to determine which classifiers to use to pare down the public set of Python systems into a good collection for further analysis \cite{omari:2018}. In addition, we have used the work they have provided to select appropriate Python systems for review by collecting meta-data on these code systems.

Selection criteria employed by Omari and Martinez included ``popular projects with long development history and multiple release cycles \cite{omari:2018}.'' All projects are open source, enabling other researchers to access the defined set. Additionally, they captured meta-data used to define our subset of repositories for our particular focus.

From their subset of repositories, we reviewed current Pylint scores from each of the 132 systems. This set gives us a sampling of data that we can now dig deeper into, comparing similar systems (similar size, a similar number of contributors, and more) and their evolution process.

\section{Filtered Respository Set} \label{sectionFilteredSet}

Once we had a narrowed set of projects from GitHub that were primarily written in Python, we culled the set more using several criteria:

\vspace{0.25cm}
\begin{enumerate}
    \item Projects that are at least 80\% Python
    \item Projects with a long history of commits
    \item Projects with large development teams (community of contributors)
    \item Projects with many releases
    \item Projects of a substantial age
\end{enumerate}
\vspace{0.25cm}

Armed with this list, we were able to use the metadata from GitHub for each of our repositories already collected and determine a cross-section of these criteria that would result in about 50 repositories for futher study.

Beginning with the languages field from GitHub, we could easily narrow down projects that had at least 80\% of the code in Python. In our set, 103 repositories contained 80\% or more Python code.

With this narrowed set, we then looked to see at which percentile all the remaining criteria would yield the desired number of repositories. We determined that using the value at the 20th percentile in each of the above categories would yield the size set we'd need.

\begin{table}[ht]
  \centering
  \begin{tabularx}{0.8\textwidth} {
    | >{\centering\arraybackslash}X 
    | >{\centering\arraybackslash}X |
  }
    \hline
      Criteria & 20th Percentile Value \\ 
    \hline\hline
      Number of Commits & 2,968 \\
      Number of Contributors & 90 \\
      Number of Releases & 44 \\
      Age (in months) & 66.4 \\
    \hline
  \end{tabularx}
  \caption{Criteria used to filter down the initial set of repositories.}
  \label{table:repositoryPercentiles}
\end{table}

Table \ref{table:repositoryPercentiles} shows the values found for each of our criteria. Using these values as our minimum requirements, we can narrow our repository set to 46 repositories (see Table \ref{table:repositorySet}).

\begin{table}[ht]
  \centering
  \begin{tabularx}{1.0\textwidth} {
    | >{\centering\arraybackslash}X |
  }
    \hline
      Respository Names \\ 
    \hline\hline
      ansible
      astropy
      autobahn-python
      aws-cli
      beets
      biopython
      boto
      buildbot
      celery
      cobbler
      conda
      cython
      django
      django-rest-framework
      electrum
      fail2ban
      gensim
      luigi
      matplotlib
      mongoengine
      mitmproxy
      mongo-python-driver
      mopidy
      networkx
      paramiko
      nova
      numba
      pandas
      peewee
      pelican
      pip
      pyramid
      ranger
      raven-python
      salt
      scikit-image
      scikit-learn
      scrapy
      sentry
      sqlalchemy
      swift
      sympy
      tornado
      web2py
      werkzeug
      youtube-dl \\
    \hline
  \end{tabularx}
  \caption{List of the 46 repositories for research focus.}
  \label{table:repositorySet}
\end{table}


% Chapter four is what you found.
\newpage
\chapter{Results} \label{chapterResults}

% Results, findings, discussion of results OR manuscripts.  It is best to also reiterate information in your literature review to help substantiate the findings of your research.

With our narrowed focus of repositories, we can now spend some more efforts looking through the data that can be collected. Specifically, we gathered data on the Refactor Warnings provided by Pylint. In Table \ref{table:smallRefactorWarningTotals} we have a shortened listing of the repositories from our set with their total count of refactor warning messages, in addition to the most commonly appearing refactor warning message. For the entire set, see Table \ref{table:allRefactorWarningTotals} in the Appendix.

\begin{table}[ht]
  \small
  \centering
  \begin{tabularx}{0.8\textwidth} {
    | l 
    | c
    | >{\centering\arraybackslash}X 
    | c |
  }
    \hline
    Repository Name & Msg Count & Top Msg & Top Msg Count \\
    \hline\hline
    sympy & 14,206 & too-many-arguments & 6,601 (46\%) \\ \hline
    ansible &  10,431 & no-else-return & 1,711 (16\%) \\ \hline
    salt &  7,814 & too-many-arguments & 1,640 (21\%) \\ \hline
    -- & -- & -- & -- \\ \hline
    ranger & 109 & no-else-return & 29 (27\%) \\ \hline
    sentry & 66 & no-self-use & 32 (48\%) \\ \hline
    raven-python & 20 & too-few-public-methods & 7 (35\%) \\ \hline
  \end{tabularx}
  \caption{Total refactor messages per repository for the 3 best and 3 worst offendors. Also provided with the refactor message that had the most warnings and its total count (and the percentage of that message from the total refactor warnings for that repository).}
  \label{table:smallRefactorWarningTotals}
\end{table}

We can see that the ``worst'' project in our set, in regards to the number of refactor warning messages provided, was the repository Sympy \cite{data:sympy}, with 14,206 refactor messages. However, project size may also weigh into how many warnings and errors are present, as larger projects are presumed to have more errors.

If we instead look at the ratio of refactor warnings to the total lines of source code, we get a very different picture, provided in Table \ref{table:smallRefactorSLOCRatio}. In this case, the repository Raven-Python \cite{data:raven-python} (which was the ``best'' in regards to refactor message count) is the ``worst'' with the highest ratio of refactoring warnings compared to the lines of source code in the project. Raven-Python also has the smallest count of SLOC in the entire repository set!

\begin{table}[ht]
  \small
  \centering
  \begin{tabularx}{1.0\textwidth} {
    | l 
    | r
    | r
    | >{\centering\arraybackslash}X |
  }
    \hline
    Repository & Total Refactor Msgs & Total Project SLOC & Ratio \\
    \hline\hline
    raven-python & 20 & 1,474 & 1.35 \\ \hline
    scrapy & 381 & 58,768 & 0.64 \\ \hline
    numba & 126 & 20,192 & 0.62 \\ \hline
    -- & -- & -- & -- \\ \hline
    electrum & 722 & 425,576 & 0.16 \\ \hline
    youtube-dl & 1,003 & 667,075 & 0.15 \\ \hline
    cython & 1,718 & 1,183,863 & 0.14 \\ \hline
  \end{tabularx}
  \caption{Total refactor messages per repository, total project source line of code (SLOC), and the ratio of refactor messages to SLOC.}
  \label{table:smallRefactorSLOCRatio}
\end{table}

Given that we now have an idea of which projects have the most refactor messages per lines of source code (``worst'' offenders) and the projects with the least refactor messages (``best'' projects for maintainability), we can look a little closer at the messages that are most frequent. Common to all six of these repositories at a high frequency are the message ``no-self-use'' and ``no-else-return''. The message ``no-self-use'' means that `self' is used as an argument but is not used in the method and should be handled differently. The message ``no-else-return'' highlights when an unnecessary block of code follows an if-conditional.

Some of the other common messages call out the use of ``too many'' of different types of objects. For example, ``too-many-branches'', ``too-many-arguments'', ``too-many-locals'', ``too-many-statements''... 

\vspace{0.25cm}
\begin{displayquote}
  ``Software maintainability these days has become one of the essential external attributes of software, which further forms a basis of research for many researchers working in the fields related to software engineering. Software maintainability can be described as the extent to which a particular software system can be changed concerning the number of Lines of Code (LOC).'' \cite{gupta:2021}
\end{displayquote}
\vspace{0.25cm}

Now, it would be helpful to understand where we stand on the Maintainability Index (MI) for these projects, as this is our form of measurement that will help us understand how the code might be able to evolve over time. The MI is calculated on a per-module basis, but for the sake of conversation (and acknowledging that we will lose some nuances by reducing it this way), let's find the average MI at the project level and add this to our table (Table \ref{table:smallRefactorSLOCRatio2}).

\begin{table}[ht]
  \small
  \centering
  \begin{tabularx}{1.0\textwidth} {
    | l 
    | r
    | r
    | r
    | >{\centering\arraybackslash}X |
  }
    \hline
    Repo & Refactor Msgs & Project SLOC & Ratio & Avg Project MI \\
    \hline\hline
    raven-python & 20 & 1474 & 1.35 & 87.02 \\ \hline
    scrapy & 381 & 58768 & 0.64 & 64.47 \\ \hline
    numba & 126 & 20192 & 0.62 & 62.55 \\ \hline
    -- & -- & -- & -- & -- \\ \hline
    electrum & 722 & 425576 & 0.16 & 39.41 \\ \hline
    youtube-dl & 1003 & 667075 & 0.15 & 54.16 \\ \hline
    cython & 1718 & 1183863 & 0.14 & 31.02 \\ \hline
  \end{tabularx}
  \caption{Total refactor messages per repository, total project source line of code (SLOC), the ratio of refactor messages to SLOC, average Maintainability Index (MI).}
  \label{table:smallRefactorSLOCRatio2}
\end{table}

When reviewing the MI for our repositories at either end of the spectrum and in context with the entire set (see Appendex \ref{table:allRefactorSLOCRatio}), we'll find that while Raven-Python has the highest number of refactor warnings when compared to the SLOC count, it also has the second highest average MI across all of its project files.

Of the full set, our highest average MI is 87.38 (belonging to Sentry) and our lowest average MI is 28.85 (belonging to MatPlotLib). Regardless, all of these average values are above a score of 20, which is Radon's lowest score provided for an ``A'' rank, that is, what would be considered a project ``very high'' maintainability.


% Chapter five is what it all means – putting the pieces together, (what’s your contribution to the research field).
\newpage
\chapter{Conclusions and Recommendations} \label{chapterConclusion}

\todo{TODO: Chapter 5 - what it all means, putting the pieces together (what is my contribution to the research field)}

% This chapter could also be called “Conclusions and Recommendations” or “Conclusions and Implications.” In general, there should be no new information presented here.  It should be a synthesis of information that you’ve already discussed. 

By collecting data and drawing our conclusions from it, with help from the insights from the studies done before ours, we may better understand metrics that can be useful regarding maintainability. Good projects will inevitably continue to grow and evolve. Understanding methods to keep code refactor on a certain level makes code easy to change. We may also find that projects with worsening scores slow down with updates and have reduced engagement.

When reviewing our surface-level data with current project Refactor scores, our three worst offenders were \emph{Ansible} \cite{data:ansible}, \emph{SymPy} \cite{data:sympy}, and \emph{Salt} \cite{data:salt}. All three projects are still quite active with development despite their poor current scores. The projects have high download rates, which may be the reason for continued development despite potential difficulty in maintenance.

Projects that may be open source or have many contributors are especially vulnerable to maintainability degrading over the evolution of a project. Having a reliable metric can be very useful in programmatically avoiding code smells and keeping code in a state that is easy to manage through simple metric checks in deployment pipelines.

We can see an example of this in reviewing some current symptoms that \emph{SymPy} is experiencing, with only 72\% code coverage and a failing build (see ``Fig.~\ref{figSymPyStatus}''). Despite the engagement and continued development, we suspect that real adaptations and evolution of the software may be difficult with this code.

\begin{figure}[ht]
    \centerline{
        \includegraphics[width=1.0\columnwidth]{SymPy_status}
    }
    \caption{A snapshot of the badges from \emph{SymPy}'s repository.}
    \label{figSymPyStatus}
\end{figure}

\todo{We have further work to do in this study to gain better understanding. With a set of several ``best'' and ``worst'' Python software systems, we will look into the history of the projects' commits. It would be useful to see how the Refactor scores have changed over time, and if the rate at which changes were pushed correlated to the increase or decrease in that Refactor score.}

\todo{Additional data can be gathered from this set that may provide more insights than this first brush of the data provides us. Understanding the impact of structural quality on the evolution of a project can provide compelling perspectives.}


% =============================================================================
% =  References
% =============================================================================

\newpage
\begin{singlespace}
  % =============================================================================
% =  Bibliography and Sources
% =============================================================================

% Rename the BibTex Bibliography to be "References"
\renewcommand{\bibname}{References}

\newpage
\bibliographystyle{IEEEtran}
\bibliography{bibliography}

\end{singlespace}

% =============================================================================
% =  Appendices
% =============================================================================

\newpage
\setcounter{chapter}{7}

\chapter*{Appendix} \label{chapterAppendix}
\addcontentsline{toc}{chapter}{Appendix}

\section{Pylint: Refactor Scores} \label{appendixPylintRefactor}

The score in Pylint is a value out of 10, with 10 being the best.

Pylint has a number of types of messages:
\begin{singlespace}
  \begin{enumerate}
    \item Convention
    \item Error
    \item Fatal
    \item Information
    \item Refactor
    \item Warning
  \end{enumerate}
\end{singlespace}

There is a very large list of all messages at \href{https://pylint.pycqa.org/en/latest/messages/messages_list.html}{Overview of All Pylint Messages}.

We spend most of our time reviewing the Refactor messages, found at \href{https://pylint.pycqa.org/en/latest/messages/messages_list.html#refactor}{Pylint Refactor Messages}.

Additionally, the Convention messages are also helpful for review, found at \href{https://pylint.pycqa.org/en/latest/messages/messages_list.html#convention}{Pylint Convention Messages}.

\section{Radon: Refactor Scores} \label{appendixRadonRefactor}

In this paper, we use Radon as one method for finding the Maintainability Index (MI) for projects, at the module (file) level.

\begin{center}
  \begin{tabular}{ c c c }
    MI Score & Rank & Maintainability \\ \hline\hline
    100 - 20 & A & Very high \\ \hline
    19 - 10 & B & Medium \\ \hline
    9 - 0 & C & Extremely Low \\ \hline    
  \end{tabular}
\end{center}


% =============================================================================
% =  Author Bibliography (Optional)
% =============================================================================

\newpage
% \begin{center}
  \large
  \textbf{BIBLIOGRAPHY}  % \footnote{IF NECESSARY (should not exceed one page except for PhDs)}
  
  
  \vspace{0.5cm}
  \textbf{\paperAuthor}

  \vspace{0.5cm}
  Candidate for the Degree of

  \vspace{0.5cm}
  Master of Science

\end{center}

\vspace{1cm}
Thesis: \MakeUppercase{\paperTitle}

\vspace{0.5cm}
Major Field: \authorConcentration

\vspace{0.5cm}
Biographical: \todo{fill something here}

\vspace{0.5cm}
Personal Data: \todo{fill something here}

\vspace{0.5cm}
Education: \todo{prior degrees}

\vspace{0.5cm}
``\paperAuthor'' has completed the requirements for the Master of Science in Computer Science at Lewis University, Romeoville, Illinois, in ``MAY'', ``2022''.

\vspace{1cm}
\begin{tabular}{ p{1.0\textwidth} } 
  \hline
  ADVISER'S APPROVAL: Dr. Mahmood Al-khassaweneh \\ 
\end{tabular}


% =============================================================================
% =  END OF THE DOCUMENT
% =============================================================================

\end{document}
