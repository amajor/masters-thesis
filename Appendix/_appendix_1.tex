\setcounter{chapter}{7}

\chapter*{Appendix} \label{chapterAppendix}
\addcontentsline{toc}{chapter}{Appendix}

\section{Pylint: Refactor Scores} \label{appendixPylintRefactor}

The score in Pylint is a value out of 10, with 10 being the best. Pylint has a number of types of messages:
\begin{singlespace}
  \begin{enumerate}
    \item Convention
    \item Error
    \item Fatal
    \item Information
    \item Refactor
    \item Warning
  \end{enumerate}
\end{singlespace}

There is a very large list of all messages at \href{https://pylint.pycqa.org/en/latest/messages/messages_list.html}{Overview of All Pylint Messages}.

We spend most of our time reviewing the Refactor messages, found at \href{https://pylint.pycqa.org/en/latest/messages/messages_list.html#refactor}{Pylint Refactor Messages}.

Additionally, the Convention messages are also helpful for review, found at \href{https://pylint.pycqa.org/en/latest/messages/messages_list.html#convention}{Pylint Convention Messages}.

\section{Radon: Maintainability Index} \label{appendixRadonRefactor}

In this paper, we use Radon as one method for finding the Maintainability Index (MI) for projects, at the module (file) level. This grading scale is provided by Radon to align with its scores \cite{radon:docs}.

\begin{center}
  \begin{tabular}{ c c c }
    MI Score & Rank & Maintainability \\ \hline\hline
    100 - 20 & A & Very high \\ \hline
    19 - 10 & B & Medium \\ \hline
    9 - 0 & C & Extremely Low \\ \hline    
  \end{tabular}
\end{center}
